\documentclass[spanish]{article}
\def\spanishoptions{mexico}
\usepackage[spanish]{babel}      % Para procesar adecuadamente el español
\usepackage[utf8]{inputenc}      % Codificación de entrada UTF-8
\usepackage[T1]{fontenc}         % Codificación de salida T1
\usepackage[official]{eurosym}   % Símbolo de euro
\usepackage{siunitx}             % Paquete para unidades métricas
\usepackage{textcomp}            % Tipografías compañeras
\usepackage{lmodern}             % Tipografía Latin Modern
\usepackage{moreverb}            % Para escribir código
\usepackage{tfrupee}
\usepackage{fontawesome5}
\usepackage{parskip}
\usepackage{hyperref}
\usepackage{tasks}
\usepackage{exsheets}
\usepackage{enumitem}
\usepackage{lastpage}
\usepackage{booktabs}
\usepackage{graphicx}
\usepackage{apalike}
\usepackage{relsize}

% Formato global:
\setlength{\parindent}{0cm}      % Párrafos sin sangría inicial
\hypersetup{pdfborder={0 0 0}}   % Ligas sin borde

\begin {document}
\title {Título}
\author {Autor}
\date {Fecha de hoy}
\maketitle

\section {Introducción}

Breve recuento acerca de la teoría del tema actual \cite{fowler1998366}.




\section {Cálculo en Python del tema actual}


\subsection {Requisitos}

\begin {itemize}
\item Restricción del problema 1
\item Restricción del problema 2
\end {itemize}


\subsection {Procedimiento}

Breve explicación del Procedimiento


\subsection {Respuestas}

Respuestas de los ejercicios



\clearpage
\section*{Código Python}
\begin{verbatim}
#!/usr/bin/env python3

# Adjuntar aquí el código de Python usado para resolver el ejercicio
\end{verbatim}

\bibliographystyle{apalike}
% \bibliographystyle{inlinebib}
\clearpage
\bibliography{../bibliografias}

\end{document}
