\documentclass[spanish]{article}
\def\spanishoptions{mexico}
\usepackage[spanish]{babel}      % Para procesar adecuadamente el español
\usepackage[utf8]{inputenc}      % Codificación de entrada UTF-8
\usepackage[T1]{fontenc}         % Codificación de salida T1
\usepackage[official]{eurosym}   % Símbolo de euro
\usepackage{siunitx}             % Paquete para unidades métricas
\usepackage{textcomp}            % Tipografías compañeras
\usepackage{lmodern}             % Tipografía Latin Modern
\usepackage{moreverb}            % Para escribir código
\usepackage{tfrupee}
\usepackage{fontawesome5}
\usepackage{parskip}
\usepackage{hyperref}
\usepackage{tasks}
\usepackage{exsheets}
\usepackage{enumitem}
\usepackage{lastpage}
\usepackage{booktabs}
\usepackage{graphicx}
\usepackage{apalike}
\usepackage{relsize}
\usepackage{listings}

% Formato global:
\setlength{\parindent}{0cm}      % Párrafos sin sangría inicial
\hypersetup{pdfborder={0 0 0}}   % Ligas sin borde

\begin {document}
\title {Optimización convexa \\ Tarea 4: Programación por metas}
\author {Acoyani Garrido Sandoval}
\date {26 de Febrero de 2026}
\maketitle

\section {Introducción}

La \textbf{programación por metas} es un método especial de la programación lineal cuyo objetivo es encontrar la forma más óptima de reconciliar metas contradictorias o que de alguna forma compiten entre sí. Este método es especialmente atractivo para la vida real, pues con frecuencia ocurren situaciones donde las restricciones de un escenario son imposibles de cumplir en su totalidad.

La idea principal detrás de este método consiste en \textbf{definir metas para los objetivos}, y posteriormente \textbf{minimizar la desviación respecto a esas metas.} Esto se hace convirtiendo nuestras restricciones en igualdades, y añadiendo 2 variables de desviación a nuestra función objetivo. Como ejemplo, si tenemos una restricción como la siguiente:

\[
   25x + 20y \le 5000
\]

Entonces convertimos nuestra desigualdad en igualdad y añadimos 2 variables adicionales, $d^-$ y $d^+$:

\[
   25x + 20y + d^- - d^+ = 5000
\]

donde $d^-$ y $d^+$ representan significativamente un \textbf{déficit} (qué tanto nuestra solución estará por debajo de la meta) y un \textbf{exceso} (qué tanto nuestra solución estará por encima de la meta), y su función matemática es absorber la diferencia entre lo que deseamos y lo que pudimos encontrar en la vida real. Dichas variables tienen 2 restricciones:

\begin{itemize}
   \item \textbf{Ambas variables deben ser positivas}, ya que no hay una sola variable que indique déficit o exceso, sino una para cada rubro.
   \item \textbf{Sólo una puede ser mayor que 0 a la vez}, ya que es imposible tener déficit y exceso al mismo tiempo.
\end{itemize}

Esta conversión de desigualdad a igualdad con exceso y déficit la hacemos sobre la restricción que podamos identificar en nuestro problema como \textbf{restricción suave,} por motivo de que hay un cierto grado de libertad dependiente de la persona a la cual aplique el problema para tener déficit o superávit en el cumplimiento de nuestra meta. Como deseamos cumplir esa meta lo mejor posible, \textbf{a partir de esta meta será nuestra función objetivo,} de la siguiente forma, siguiendo el ejemplo de la ecuación anterior:

\[
   25x + 20y + \rho d^- - \tau d^+ = 5000
\]

donde $\rho$ y $\tau$ son \textbf{factores de penalización} para nuestro déficit y superávit que se determinan al momento de resolver el problema con el fin de establecer qué tan grave es el déficit y el superávit: entre más alto sea ese parámetro, la ejecución de la solución tenderá fuertemente a evitar incurrir en déficit o superávit.

Y entonces, el problema consistirá en \textbf{minimizar lo que queremos evitar en nuestro problema}. Si nuestra meta suave es evitar incurrir en exceso, \textbf{minimizamos el exceso $d^+$}. Si nuestra meta suave es evitar el déficit, minimizamos el déficit $d^-$. Y si nuestra meta suave es ajustarnos lo más posible a un valor fijo, \textbf{minimizamos ambas variables.}

Entonces, suponiendo que nuestro problema consiste en evitar exceder un presupuesto establecido, \textbf{nuestra función objetivo} es:

\[
   min \: Z = 25x + 20y + \rho d^+
\]

donde ya no figura la cantidad de 5000 ni el déficit $d^-$ por motivo de que dichas cantidades son parte de la meta, no de nuestro problema, el cual es minimizar el exceso en el cumplimiento de la meta.



\section {Escenario}

Travel Tech es una agencia de viajes corporativos. Un día, la empresa recibe una petición para un viaje grande: el retiro anual de una startup tecnológica en la Ciudad de México, para lo cual será necesario alojar a 50 empleados durante un fin de semana. El cliente manifiesta que no desea hoteles convencionales, y en su lugar preferiría \faAirbnb{} Airbnb's.

Sin embargo, algunos de sus requisitos resultan ser conflictivos, por lo que hay que plantear una propuesta que equilibre lo mejor posible sus demandas:

\begin {enumerate}
\item \textbf{Logística:} Prefieren pocas casas grandes, con el fin de minimizar la cantidad de reservaciones.
\item \textbf{Presupuesto:} Hay un tope de gastos, pero hay una cierta flexibilidad para excederlo.
\item \textbf{Calidad:} Sólo aceptan lugares bien calificados.
\end {enumerate}

Para ello, vamos a construir un modelo de optimización que seleccione la combinación ideal de propiedades, cumpliendo los requisitos físicos (restricciones duras) y minimizando las desviaciones de las metas financieras (restricciones suaves).



\section{Conjunto de datos}

La empresa \faAirbnb{} Airbnb pone a disposición del público un conjunto de datos en su portal Inside Airbnb \cite{airbnb2026} que muestra datos trimestrales al día de hoy actualizados al dís 27 de Septiembre de 2025. Dicho conjunto de datos consta de una gran cantidad de inmuebles que han sido listados en la plataforma, y contiene datos tales como la colonia donde se ubica, la calificación del inmueble, el cupo máximo, o el precio en pesos; con lo que el conjunto de datos tiene información suficiente para poder tomar una decisión.

Sin embargo, antes de poder usarlo, es necesario limpiar el conjunto de datos. Esto se logra mediante el siguiente código:



\section{Planteamiento matemático}




\section{Solución del problema}



\section{Resultados encontrados en el conjunto de datos}




\section {Análisis de sensibilidad}


\subsection{Rangos de factibilidad}


\subsection{Rangos de optimalidad}


\subsection{Precios sombra}



\section{Diferentes escenarios}


\subsection{Escenario austero}


\subsection{Escenario de influencer}


\subsection{Escenario exigente}


\subsection{Escenario familiar}


\subsection{Escenario de cercanía}




\clearpage
\section*{Código Python}
\begin{verbatim}
#!/usr/bin/env python3

# Adjuntar aquí el código de Python usado para resolver el ejercicio
\end{verbatim}

\bibliographystyle{apalike}
% \bibliographystyle{inlinebib}
\clearpage
\bibliography{../bibliografias}

\end{document}
