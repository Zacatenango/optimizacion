\documentclass[spanish]{article}
\def\spanishoptions{mexico}
\usepackage[spanish]{babel}      % Para procesar adecuadamente el español
\usepackage[utf8]{inputenc}      % Codificación de entrada UTF-8
\usepackage[T1]{fontenc}         % Codificación de salida T1
\usepackage[official]{eurosym}   % Símbolo de euro
\usepackage{siunitx}             % Paquete para unidades métricas
\usepackage{textcomp}            % Tipografías compañeras
\usepackage{lmodern}             % Tipografía Latin Modern
\usepackage{moreverb}            % Para escribir código
\usepackage{tfrupee}
\usepackage{fontawesome5}
\usepackage{parskip}
\usepackage{hyperref}
\usepackage{tasks}
\usepackage{exsheets}
\usepackage{enumitem}
\usepackage{lastpage}
\usepackage{booktabs}
\usepackage{graphicx}
\usepackage{apalike}
\usepackage{relsize}

% Formato global:
\setlength{\parindent}{0cm}      % Párrafos sin sangría inicial
\hypersetup{pdfborder={0 0 0}}   % Ligas sin borde

\begin {document}
\title {Optimización convexa \\ Tarea 4: Programación por metas}
\author {Acoyani Garrido Sandoval}
\date {26 de Febrero de 2026}
\maketitle

\section {Introducción}

Breve recuento acerca de la teoría del tema actual \cite{fowler1998366}.




\section {Escenario}

Travel Tech es una agencia de viajes corporativos. Un día, la empresa recibe una petición para un viaje grande: el retiro anual de una startup tecnológica en la Ciudad de México, para lo cual será necesario alojar a 50 empleados durante un fin de semana. El cliente manifiesta que no desea hoteles convencionales, y en su lugar preferiría \faAirbnb{} Airbnb's.

Sin embargo, algunos de sus requisitos resultan ser conflictivos, por lo que hay que plantear una propuesta que equilibre lo mejor posible sus demandas:

\begin {enumerate}
\item \textbf{Logística:} Prefieren pocas casas grandes, con el fin de minimizar la cantidad de reservaciones.
\item \textbf{Presupuesto:} Hay un tope de gastos, pero hay una cierta flexibilidad para excederlo.
\item \textbf{Calidad:} Sólo aceptan lugares bien calificados.
\end {enumerate}

Para ello, vamos a construir un modelo de optimización que seleccione la combinación ideal de propiedades, cumpliendo los requisitos físicos (restricciones duras) y minimizando las desviaciones de las metas financieras (restricciones suaves).



\section{Conjunto de datos}

La empresa \faAirbnb{} Airbnb pone a disposición del público un conjunto de datos en su portal Inside Airbnb \cite{airbnb2026} que muestra datos trimestrales al día de hoy actualizados al dís 27 de Septiembre de 2025. Dicho conjunto de datos consta de una gran cantidad de inmuebles que han sido listados en la plataforma, y contiene datos tales como la colonia donde se ubica, la calificación del inmueble, el cupo máximo, o el precio en pesos; con lo que el conjunto de datos tiene información suficiente para poder tomar una decisión.

Sin embargo, antes de poder usarlo, es necesario limpiar el conjunto de datos. Esto se logra mediante:



\section{Planteamiento matemático}




\section{Solución del problema}



\section{Resultados encontrados en el conjunto de datos}




\section {Análisis de sensibilidad}


\subsection{Rangos de factibilidad}


\subsection{Rangos de optimalidad}


\subsection{Precios sombra}



\section{Diferentes escenarios}


\subsection{Escenario austero}


\subsection{Escenario de influencer}


\subsection{Escenario exigente}


\subsection{Escenario familiar}


\subsection{Escenario de cercanía}




\clearpage
\section*{Código Python}
\begin{verbatim}
#!/usr/bin/env python3

# Adjuntar aquí el código de Python usado para resolver el ejercicio
\end{verbatim}

\bibliographystyle{apalike}
% \bibliographystyle{inlinebib}
\clearpage
\bibliography{../bibliografias}

\end{document}
